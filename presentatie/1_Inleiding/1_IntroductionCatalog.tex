%% !TeX program = pdflatex
\documentclass[../presentatie.tex]{subfiles}

\csname inputexit\endcsname
\csname inputentrance\endcsname
\def\named#1{}

\begin{document}
    \clearrecentlist

    \inputentrance\named{intro-wordcomp-doc}
    \begin{frame}
        \frametitle{\LaTeX{} vs Word}

        \begin{columns}
            \begin{column}{0.5\textwidth}
                \includegraphics[width=\linewidth,height=0.8\textheight,keepaspectratio]{assets/1_Inleiding/basicDocWordSnippet.png}
            \end{column}
            \begin{column}{0.5\textwidth}
                \includegraphics[width=\linewidth,height=0.8\textheight,keepaspectratio]{assets/1_Inleiding/basicDocLaTeXSnippet.png}
            \end{column}
        \end{columns}
    \end{frame}

    \inputentrance\named{intro-wordcomp-doc-eqfragment}
    \begin{frame}
        \frametitle{\LaTeX{} vs Word}

        \bgroup
        \setlength{\fboxsep}{0pt}
        \fbox{\includegraphics[width=\linewidth,height=0.5\textheight,keepaspectratio]{assets/1_Inleiding/basicDocWordSnippetEq.png}}
        \medskip

        \fbox{\includegraphics[width=\linewidth,height=0.5\textheight,keepaspectratio]{assets/1_Inleiding/basicDocLaTeXSnippetEq.png}}
        \egroup
    \end{frame}

    \let\frameselection\somethingundefined

    \inputentrance\named{intro-wordcomp-ondermotorkap}
    \ifx\frameselection\somethingundefined
        \def\frameselection{1-2}
    \fi

    \begin{saveblock}{basicDocCode}
        \begin{highlightblock}[linewidth=\textwidth,gobble=12]
            \title{My document}
            \author{Vincent Kuhlmann}
            \date{3 May 2021}
            
            \begin{document}
            \maketitle        
            \section{Lorem ipsum}
            Lorem ipsum dolor sit amet, consectetuer ...
    
            \begin{align}
                f(x) = \dfrac{1}{\sigma\sqrt{2\pi}} e^{
                    -\frac{1}{2}\left(\frac{x-\mu}{\sigma}\right)^2}
            \end{align}
        \end{highlightblock}
    \end{saveblock}

    \ifx\frameselection\somethingundefined
        \def\frameselection{1-}
    \fi

    \def\pasteframeselection#1{\begin{frame}<#1>}%
    \expandafter\pasteframeselection\expandafter{\frameselection}%
        \frametitle{\LaTeX{} vs Word}

        \ifbool{english}{
            Inner workings: big difference.

            Word: Edit visually,\\\LaTeX: Edit code (text)
        }{
            Onder de motorkap: groot verschil.
            
            Word: Visueel, \LaTeX: Code (tekst).
        }

        \pause
        \medskip

        \begin{columns}[t]%
            \begin{column}{0.6\textwidth}
                \adjustbox{max height=0.6\textheight}{
                    \useblock{basicDocCode}
                }
            \end{column}%
            \begin{column}{0.4\textwidth}%
                \adjustbox{fbox=0.5pt 0pt 0pt,margin=-30pt 0pt 0pt 0pt,set height=0pt}{%
                    \includegraphics[width=1.6\linewidth,height=0.9\textheight,keepaspectratio]{assets/1_Inleiding/basicDocLaTeXSnippet.png}%
                }
            \end{column}%
        \end{columns}
    \end{frame}

    \let\frameselection\somethingundefined
    
    \inputentrance\named{intro-visueelcomp}
    
    \ifx\frameselection\somethingundefined
        \def\frameselection{1-4}
    \fi

    \def\pasteframeselection#1{\begin{frame}<#1>}%
    \expandafter\pasteframeselection\expandafter{\frameselection}%
    [label=codeoverview]
        \frametitle{Code vs \lang,Visual,Visueel,}

        \parbox[c][0.8\textheight]{0.35\textwidth}{
            \begin{itemize}
                \item<1-> Websites \& Apps
                \par\uncover<3->{
                    \textbf{Complex}
                }
                
                \item<4-> Wikipedia
                \par\uncover<5->{
                    \textbf{Consistent}
                }

                \item<6-> WhatsApp
                \par\uncover<7->{
                    \textbf{\lang,Expandable,Uitbreidbaar,}
                }
            \end{itemize}
        }\hfil
        %\parbox[c]{1pt}{\textcolor{red}{\rule{0.5pt}{0.3\textheight}}}%
        \hfil
        \parbox[c]{0.6\textwidth}{
            \unless\ifishandout
            \only<1>{
                \includegraphics[
                    width=\linewidth,height=0.8\textheight,keepaspectratio
                ]{assets/1_Inleiding/websiteVisual.png}
            }
            \only<2-3>{
                \adjustbox{
                    trim=0pt {0.5\height} {0.5\width} 0pt,clip,width=\linewidth
                }{%
                    \includegraphics[
                        width=\linewidth,height=0.8\textheight,keepaspectratio
                    ]{assets/1_Inleiding/websiteVisual.png}%
                }
            }
            \only<4-5>{
                \centering
                \includegraphics[width=\linewidth,height=0.8\textheight,keepaspectratio]{assets/1_Inleiding/wikipediaVisual}
                %\includegraphics[width=\linewidth,height=0.8\textheight,keepaspectratio]{assets/wikipediaVisual2}
            }
            \fi
            \only<6-7>{
                \includegraphics[width=\linewidth]{assets/1_Inleiding/whatsappStyles2.png}
                
                \bigskip
                
                %\includegraphics[width=\linewidth]{whatsappStylesResultCropped.jpg}
                \includegraphics[width=\linewidth]{assets/1_Inleiding/whatsappStylesResult2.png}
            }
        }
    \end{frame}

    \inputentrance\named{intro-visueelcomp-wiki-example}

    \begin{saveblock}{ninglinspoFragment1}
        \begin{Verbatim}[tabsize=4,gobble=8]
        {{Infobox rivier
                | naam          = Ninglinspo
                | afbeelding    = Ninglinspo - arrivée d
                | onderschrift  = De Ninglinspo niet ver
                | lengte        = 15
                | hoogte        = 420
                | hoogtemonding = 270
                | verhang       = 
                | debiet        = 
        \end{Verbatim}
    \end{saveblock}

    \begin{saveblock}{ninglinspoFragment2}
        \begin{Verbatim}[tabsize=4,gobble=8]
        De oorspronkelijke naam is eigenlijk de "Doulneu
        een Els. Er werd reeds gesproken over de rivier
        charter van [[Sigibert III]].
        <ref>informatiebord aan de monding van de Ningli
        \end{Verbatim}
    \end{saveblock}

    \begin{frame}
        \frametitle{Code vs \lang,Visual,Visueel,}

        % \begin{columns}
        %    \begin{column}{0.45\textwidth}
        %\centering
        \adjustbox{
            trim=0pt 0pt 5em 0pt,
            max height=0.9\textheight,max width=0.5\linewidth,
            %fbox=1pt 0pt 2pt,
            cframe=green!50!white 0.5pt 0pt 2pt,
            valign=M
        }{
            \useblock{ninglinspoFragment1}
        }%
        %\hspace{2pt}%
        \adjustbox{
            cframe=blue!50!white 0.5pt 0pt 0pt,
            valign=M
        }{%
            \includegraphics[
                trim=7pt 0pt 7pt 0pt,clip,
                height=0.5\textheight,width=0.5\linewidth,keepaspectratio]{assets/1_Inleiding/wikipediaInfobox.png}%
        }
        \vspace{5pt}

        \adjustbox{
            trim=0pt 0pt 5em 0pt,
            max height=0.9\textheight, max width=0.5\linewidth,
            %fbox=1pt 0pt 2pt,
            cframe=green!50!white 0.5pt 0pt 2pt,
            valign=M
        }{
            \useblock{ninglinspoFragment2}
        }%
        \adjustbox{
            cframe=blue!50!white 0.5pt 0pt 0pt,
            valign=M
        }{%
            \includegraphics[height=0.5\textheight,width=0.5\linewidth,keepaspectratio]{assets/1_Inleiding/wikipediaHyperAndFootnotes.png}%
        }
    \end{frame}

    \againframe<5>{codeoverview}

    \inputentrance\named{intro-visueelcomp-lemmaproof}

    \begin{saveblock}{repeatel}
        \begin{highlightblock}[linewidth=0.5\textwidth,gobble=12]
            \begin{lemma}
                Lorem ipsum dolor sit
                ... eget dolor.
                
                \begin{proof}
                    Aenean massa. Cum
                    ... quis enim.
                \end{proof}
            \end{lemma}
        \end{highlightblock}
    \end{saveblock}

    \begin{frame}
        \frametitle{Code vs \lang,Visual,Visueel,}
        \begin{columns}
            \begin{column}{0.5\textwidth}
                \useblock{repeatel}
            \end{column}
            \begin{column}{0.5\textwidth}
                \includegraphics[width=\linewidth,height=0.8\textheight,keepaspectratio]{assets/1_Inleiding/latexRepeatEl.pdf}
            \end{column}
        \end{columns}
    \end{frame}

    \let\frameselection\somethingundefined
    
    \inputentrance\named{intro-visueelcomp-wordugly}
    
    \ifx\frameselection\somethingundefined
        \def\frameselection{1-}
    \fi
    
    \def\pasteframeselection#1{\begin{frame}<#1>}%
    \expandafter\pasteframeselection\expandafter{\frameselection}%
        \frametitle{Code vs Visueel}
    
        \only<-1>{\includegraphics[width=\linewidth,height=0.8\textheight,keepaspectratio]{assets/1_Inleiding/wordkader2.png}}
        \unless\ifishandout
        \only<2->{\includegraphics[width=\linewidth,height=0.8\textheight,keepaspectratio]{assets/1_Inleiding/wordkaderRand.png}}
        \fi
    
    \end{frame}

    \againframe<5->{codeoverview}

    \inputentrance\named{intro-overleaf}
    \addtorecentlist{Overleaf}
    
    \begin{frame}
        \frametitle{Overleaf}
    
        %\includegraphics[trim=10px 120px 50px 30px,clip,width=0.95\textwidth]{texnicieWebsiteOverleafTeXstudio.png}
        \lang{
            \begin{list}{}{
                \itemindent-\leftmargin
		        \setlength{\itemsep}{2pt}
            }
                \item\textbf{LaTeX} is the programming language.
                \item\textbf{Overleaf} is a website where you can write and
                    compile LaTeX.
                \item\textbf{Visual Studio Code} is a desktop app where you can
                    write and compile LaTeX.
                \item\textbf{MiKTeX} does compilation for Visual Studio code.
            \end{list}
        }{
            \includegraphics[width=\textwidth]{assets/overleafDisambiguation.png}
        }
    
        \begin{columns}
            \begin{column}{0.3\textwidth}
                \includegraphics[width=\linewidth]{assets/1_Inleiding/overleafCreateBlankProject.png}
            \end{column}
            \begin{column}{0.7\textwidth}
                \lang{%
                    For now: Overleaf.
                }{%
                    Op het einde nog woordje hierover.
                    
                    Voor nu: Overleaf.
                }
                
                \medskip
                \lang{%
                    Want VS Code? Instructions at
                    \href{https://vkuhlmann.com/latex/installation}{\nolinkurl{vkuhlmann.com/latex/installation}}
                }{%
                    Nu al niet-commerci\"ele variant installeren? \url{a-es2.nl/texnicie}
                }
            \end{column}
        \end{columns}
    \end{frame}
    
    \inputentrance\named{intro-simpledoc}
    \begin{saveblock}{baseDocument}
        \begin{highlightblock}[linewidth=0.5\textwidth,gobble=12]
            \documentclass{article}
            \usepackage[utf8]{inputenc}
    
            \title{My document}
            \author{Vincent Kuhlmann}
            \date{1 May 2021}
    
            \begin{document}
            \maketitle
            \section{Introduction}
    
            ~\lang,Hello everyone!,Hallo iedereen!,~
            \end{document}
        \end{highlightblock}
    \end{saveblock}

    \addtorecentlist{\lang,simple,simpel, document}
    
    \begin{frame}
        \frametitle{\lang,Simple,Simpel, document}
    
        \begin{columns}
            \begin{column}{0.5\textwidth}
                \useblock{baseDocument}
            \end{column}
            \begin{column}{0.5\textwidth}
                \includegraphics[width=\linewidth]{assets/1_Inleiding/basicDocumentOutput.png}
            \end{column}
        \end{columns}
    \end{frame}
\end{document}
\inputentrance
